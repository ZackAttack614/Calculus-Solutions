\documentclass{article}

\usepackage[shortlabels]{enumitem}
\usepackage{amsmath}
\usepackage{empheq}
\usepackage[most]{tcolorbox}

\graphicspath{ {./images/} }

\title{Solutions to Selected Problems in §2.1\par
Limits, Rates of Change, and Tangent Lines}
% \date{November 27, 2021}
% \author{Zack}
\begin{document}
  \pagenumbering{gobble}
  \maketitle
  \newpage
  \pagenumbering{arabic}

  \section{Problem 1}
  \paragraph{Setup} A ball is dropped from a state of rest at time \( t = 0 \). The distance traveled after \( t \) seconds is \( s(t) = 16t^2 \) ft.
  \begin{enumerate}[(a)]
    \item \textbf{How far does the ball travel during the time interval [2, 2.5]?}

    \paragraph{Solution} To find how far the ball traveled in the interval, we simply find the difference in \(s\) at the beginning and end times.

    \begin{equation} \label{ballDistance1}
      \Delta{s} = s_{2} - s_{1} = s(2.5) - s(2) = 100 - 64 = \boxed{36 ft}
    \end{equation}

    \item \textbf{Compute the average velocity over [2, 2.5].}

    \paragraph{Solution} To find the average velocity of the ball, we find the slope of the secant line through \((2, s(2)) \) and \( (2.5, s(2.5)) \).

    \begin{equation}
      \overline{v} = \frac{\Delta{s}}{\Delta{t}} = \frac{s_{2} - s_{1}}{t_{2} - t_{1}} = \frac{s(2.5) - s(2)}{2.5 - 2} = \frac{100 - 64}{0.5} = \frac{36}{0.5} = \boxed{72 \frac{ft}{s}}
    \end{equation}
    \item \textbf{Compute the average velocity over time intervals [2, 2.01], [2, 2.001], [2, 2.00001]. Use this to estimate the object's instantaneous velocity at \( t = 2 \).}
    \paragraph{Solution} The average velocities at the given time intervals are as follow\dots
    \begin{center}
      \begin{tabular}{||c c c c | c||}
       \hline
       \(t_{1}\) & \(s(t_{1})\) & \(t_{2}\) & \(s(t_{2})\) & \(\overline{v}\) \\
       \hline\hline
       2 s. & 64 ft. & 2.01 s. & 64.6416 ft. & 64.16 ft/s \\
       \hline
       2 s. & 64 ft. & 2.001 s. & 64.064016 ft. & 64.016 ft/s \\
       \hline
       2 s. & 64 ft. & 2.00001 s. & 64.0006400016 ft. & 64.00016 ft/s \\
       \hline
      \end{tabular}
    \end{center}

    As \( t_{2} \) approaches the value of \( t_{1} \), we see the average velocity approach \(64 \frac{ft}{s}\). Therefore\dots

    \begin{equation}
      \boxed{v(2) \approx 64 \frac{ft}{s}}
    \end{equation}
  \end{enumerate}
  \newpage

  \section{Problem 2}
  \paragraph{Problem Statement}
  A wrench is released from a state of rest at time \(t=0\). Estimate the wrench's instantaneous velocity at \(t=1\), assuming that the distance traveled after \(t\) seconds is \(s(t)=16t^2\).

  \paragraph{Solution}
  To estimate the instantaneous velocity, we simply find the slope of a secant line going through \((1, s(1)) \) and \( (1 + \varepsilon, s(1 + \varepsilon)) \) for some sufficiently small \(\varepsilon\). I'll choose \(\varepsilon = 0.001\).

  \begin{equation}
    v \approx \frac{\Delta{s}}{\Delta{t}} = \frac{s(1 + \varepsilon) - s(1)}{(1 + \varepsilon) - 1} = \frac{16(1.001)^2 - 16(1)^2}{0.001} = \boxed{32.016 \frac{ft}{s}}
  \end{equation}

  \section{Problem 3}
  \paragraph{Problem Statement} Let \(v=20\sqrt{T}\), as in Example 2. Estimate the instantaneous ROC of \(v\) with respect to \(T\) when \(T=300 K\).

  \paragraph{Solution}
  To estimate the instantaneous ROC of \(v\), we simply find the slope of a secant line going through \((300, v(300)) \) and \( (300 + \varepsilon, v(300 + \varepsilon)) \) for some sufficiently small \(\varepsilon\). I'll choose \(\varepsilon = 0.001\).

  \begin{equation}
    \frac{\Delta{v}}{\Delta{T}} = \frac{v(300+\varepsilon) - v(300)}{(300 + \varepsilon) - 300} = \frac{20\sqrt{300.001} - 20\sqrt{300}}{0.001} = \boxed{0.577}
  \end{equation}

  \section{Problem 4}
  \paragraph{Problem Statement} Compute \(\frac{\Delta{y}}{\Delta{x}}\) for the interval [2, 5], where \(y=4x - 9\). What is the instantaneous ROC of \(y\) with respect to \(x\) at \(x=2\)?

  \paragraph{Solution} The average rate of change over the interval [2, 5] is simply the slope of the secant line going through \((2, y(2)) \) and \( (5, y(5)) \).

  \begin{equation}
    \frac{\Delta{y}}{\Delta{x}} = \frac{y(5) - y(2)}{5 - 2} = \frac{(4(5) - 9) - (4(2) - 9)}{3} = \frac{12}{3} = \boxed{4}
  \end{equation}

  To estimate the instantaneous rate of change of \(y\) at \(x=2\), we'll find the slope of the secant line going through \((2, y(2)) \) and \( (2 + \varepsilon, y(2 + \varepsilon)) \) for some sufficiently small \(\varepsilon\). I'll choose \(\varepsilon = 0.001\).

  \begin{equation}
    \frac{\Delta{y}}{\Delta{x}} = \frac{y(2 + \varepsilon) - y(2)}{(2 + \varepsilon) - 2} = \frac{(4(2 + \varepsilon) - 9) - (4(2) - 9)}{\varepsilon} = \frac{8.004 - 8}{0.001} = \boxed{4}
  \end{equation}

  \newpage
  \section{Problems 5 and 6}
  \paragraph{Setup} A stone is tossed in the air from ground level with an initial velocity of 15 m/s. Its height at time \(t\) is \(h(t) = 15t - 4.9t^2\) m.
  \paragraph{Problem 5 Statement} Compute the stone's average velocity over the time interval [0.5, 2.5] and indicate the corresponding secant line on a sketch of the graph of \(h(t)\).

  \paragraph{Solution} The average rate of change over the interval [0.5, 2.5] is simply the slope of the secant line going through \((0.5, h(0.5)) \) and \( (2.5, h(2.5)) \).

  \begin{equation}
    \frac{\Delta{h}}{\Delta{t}} = \frac{y(2.5) - y(0.5)}{2.5 - 0.5} = \frac{(15(2.5) - 4.9(2.5)^2) - (15(0.5) - 4.9(0.5)^2)}{2} = \boxed{0.3}
  \end{equation}

  \includegraphics[scale=0.7]{problem_5}

  \newpage
  \paragraph{Problem 6 Statement} Compute the stone's average velocity over the time intervals [1, 1.01], [1, 1.001], [1, 1.0001] and [0.99, 1], [0.999, 1], [0.9999, 1]. Use this to estimate the instantaneous velocity at \(t=1\).

  \paragraph{Solution} The table below shows the average velocities over the given intervals\dots
  \begin{center}
    \begin{tabular}{||c c c c | c||}
     \hline
     \(t_{1}\) & \(h(t_{1})\) & \(t_{2}\) & \(h(t_{2})\) & \(\overline{v}\) \\
     \hline\hline
     1 s & 10.1 m & 1.01 s & 10.35351 m & 5.151 m/s \\
     \hline
     1 s & 10.1 m & 1.001 s & 10.1051951 m & 5.1951 m/s \\
     \hline
     1 s & 10.1 m & 1.0001 s & 10.100519951 m & 5.19951 m/s \\
     \hline
     0.99 s & 10.04751 m & 1 s & 10.1 m & 5.249 m/s \\
     \hline
     0.999 s & 10.0947951 m & 1 s & 10.1 m & 5.2049 m/s \\
     \hline
     0.9999 s & 10.099479951 m & 1 s & 10.1 m & 5.20049 m/s \\
     \hline
    \end{tabular}
  \end{center}

  Clearly, as we get closer to \(t_{2} = t_{1}\), we are approaching the value of the instantaneous velocity\dots
  \begin{equation}
    \boxed{\overline{v} = 5.2 m/s}
  \end{equation}
  \newpage

  \section{Problem 7}
  \paragraph{Setup} With an initial deposit of \$100, the balance in a bank account after \(t\) years is \(f(t) = 100(1.08)^t\) dollars.

  \begin{enumerate}[(a)]
    \item \textbf{What are the units of the ROC of \(f(t)\)?}
    \paragraph{Solution} The rate of change is \(\frac{\Delta{f}}{\Delta{t}}\), where \([\Delta{f}] = \$\) and \([\Delta{t}] = yr\). Therefore\dots
  
    \begin{equation}
      \boxed{[\frac{\Delta{f}}{\Delta{t}}] = \frac{\$}{yr}}
    \end{equation}

    \item \textbf{Find the average ROC over [0, 0.5] and [0, 1].}
    \paragraph{Solution} The average rate of change over the given time intervals are as follow\dots

    \begin{center}
      \begin{tabular}{||c c c c | c||}
      \hline
      \(t_{1}\) & \(f(t_{1})\) & \(t_{2}\) & \(f(t_{2})\) & \(\overline{ROC}\) \\
      \hline\hline
      0 yrs. & \$100.00 & 0.5 yrs. & \$103.92 & 7.85 \(\frac{\$}{yr}\) \\
      \hline
      0 yrs. & \$100.00 & 1 yr. & \$108.00 & 8 \(\frac{\$}{yr}\) \\
      \hline
      \end{tabular}
    \end{center}

    \item \textbf{Estimate the instantaneous rate of change at \(t=0.5\) by computing the average ROC over intervals to the left and right of \(t=0.5\).}

    \paragraph{Solution} We'll compute the average rate of change of \(f(t)\) for \(t\) in the intervals [0, 0.5] and [0.5, 1], then average those out to estimate the instantaneous rate of change.

    \begin{center}
      \begin{tabular}{||c c c c | c||}
      \hline
      \(t_{1}\) & \(f(t_{1})\) & \(t_{2}\) & \(f(t_{2})\) & \(\overline{ROC}\) \\
      \hline\hline
      0 yrs. & \$100.00 & 0.5 yrs. & \$103.92 & 7.85 \(\frac{\$}{yr}\) \\
      \hline
      0.5 yrs. & \$103.92 & 1 yr. & \$108.00 & 8.15 \(\frac{\$}{yr}\) \\
      \hline
      \end{tabular}
    \end{center}

    The average between the rates of change of the two intervals is\dots

    \begin{equation}
      \frac{7.85 + 8.15}{2} = \boxed{8 \frac{\$}{yr}}
    \end{equation}
  \end{enumerate}
  \newpage

  \section{Problem 8}
  \paragraph{Problem Statement} The distance traveled by a particle at time \(t\) is \(s(t) = t^3 + t\). Compute the average velocity over the time interval [1, 4] and estimate the instantaneous velocity at \(t=1\).

  \paragraph{Solution} The average velocity over the interval [1, 4] is simply the slope of the secant line going through \((1, s(1)) \) and \( (4, s(4)) \).

  \begin{equation}
    \overline{v} = \frac{s(4) - s(1)}{4-1} = \frac{(4^3 + 4) - (1^3 + 1)}{4-1} = \frac{66}{3} = \boxed{22}
  \end{equation}

  To estimate the instantaneous velocity of the particle at \(t=1\), we'll find the slope of the secant line going through \((1, s(1)) \) and \( (1 + \varepsilon, s(1 + \varepsilon)) \) for some sufficiently small \(\varepsilon\). I'll choose \(\varepsilon = 0.001\).

  \begin{equation}
    \frac{\Delta{s}}{\Delta{t}} = \frac{s(1+\varepsilon)-s(1)}{(1+\varepsilon) - 1} = \frac{(1.001^3 + 1.001) - (1^3 + 1)}{0.001} = \frac{2.004003 - 2}{0.001} = \boxed{4.003}
  \end{equation}

  \includegraphics[scale=0.7]{problem_8}
  \newpage

  \section{Problem 9}
  \paragraph{Problem Statement} Estimate the instantaneous rate of change of \(P(x) = 4x^2 - 3\) at \(x=2\).

  \paragraph{Solution} To estimate the instantaneous rate of change of \(P(x) = 4x^2 - 3\) at \(x=2\), we'll find the slope of the secant line going through \((2, P(2)) \) and \( (2 + \varepsilon, P(2 + \varepsilon)) \) for some sufficiently small \(\varepsilon\). I'll choose \(\varepsilon = 0.001\).

  \begin{equation}
    \frac{\Delta{P}}{\Delta{x}} = \frac{P(2+\varepsilon)-P(2)}{(2+\varepsilon) - 2} = \frac{(4(2.001^2)-3) - (4(2^2)-3)}{0.001} \approx \boxed{16}
  \end{equation}

  \includegraphics[scale=0.7]{problem_9}
  \newpage

  \section{Problem 10}
  \paragraph{Problem Statement} Estimate the instantaneous rate of change of \(f(t) = 3t-5\) at \(t=-9\).

  \paragraph{Solution} To estimate the instantaneous rate of change of \(f(t) = 3t-5\) at \(t=-9\), we'll find the slope of the secant line going through \((-9, f(-9)) \) and \( (-9 + \varepsilon, f(-9 + \varepsilon)) \) for some sufficiently small \(\varepsilon\). I'll choose \(\varepsilon = 0.001\).

  \begin{equation}
    \frac{\Delta{f}}{\Delta{t}} = \frac{f(-9+\varepsilon)-f(-9)}{(-9+\varepsilon) - (-9)} = \frac{(3(-8.999)-5) - (3(-9)-5)}{0.001} = \boxed{3}
  \end{equation}

  \includegraphics[scale=0.7]{problem_10}
  \newpage

  \section{Problem 11}
  \paragraph{Problem Statement} Estimate the instantaneous rate of change of \(y(x) = \frac{1}{x+2}\) at \(x=2\).

  \paragraph{Solution} To estimate the instantaneous rate of change of \(y(x) = \frac{1}{x+2}\) at \(x=2\), we'll find the slope of the secant line going through \((2, y(2)) \) and \( (2 + \varepsilon, y(2 + \varepsilon)) \) for some sufficiently small \(\varepsilon\). I'll choose \(\varepsilon = 0.001\).

  \begin{equation}
    \frac{\Delta{y}}{\Delta{x}} = \frac{y(2+\varepsilon)-y(2)}{(2+\varepsilon) - 2} = \frac{\frac{1}{4.001} - \frac{1}{4}}{0.001} \approx \boxed{\frac{-1}{16}}
  \end{equation}

  \includegraphics[scale=0.7]{problem_11}
  \newpage

  \section{Problem 12}
  \paragraph{Problem Statement} Estimate the instantaneous rate of change of \(y(t) = \sqrt{3t + 1}\) at \(t=1\).

  \paragraph{Solution} To estimate the instantaneous rate of change of \(y(t) = \sqrt{3t + 1}\) at \(t=1\), we'll find the slope of the secant line going through \((1, y(1)) \) and \( (1 + \varepsilon, y(1 + \varepsilon)) \) for some sufficiently small \(\varepsilon\). I'll choose \(\varepsilon = 0.001\).

  \begin{equation}
    \frac{\Delta{y}}{\Delta{t}} = \frac{y(1+\varepsilon)-y(1)}{(1+\varepsilon) - 1} = \frac{\sqrt{3(1+0.001) + 1} - \sqrt{3(1) + 1}}{0.001} \approx \boxed{\frac{3}{4}}
  \end{equation}

  \includegraphics[scale=0.7]{problem_12}
  \newpage

  \section{Problem 13}
  \paragraph{Problem Statement} The atmospheric temperature \(T\) (in \(^\circ\)F) above a certain point on earth is \(T=59-0.00356h\), where \(h\) is the altitude in feet (valid for \(h \ge 37,000\)). What are the average and instantaneous rates of change of \(T\) with respect to \(h\)? Why are they the same? Sketch the graph of \(T\) for \(h \ge 37,000\).

  \paragraph{Solution} Note that the equation for \(T\) is linear with respect to \(h\), and so all tangent lines to \(T(h)\) are actually \textit{equal} to the equation \(T(h)\). Thus, it has the same slope as \(T(h)\), which is its rate of change. It is also \textit{constant} for all \(h\).

  \begin{equation}
    \overline{ROC} = \boxed{-0.00356}
  \end{equation}
  \begin{equation}
    \frac{\Delta{T}}{\Delta{h}} = \boxed{-0.00356}
  \end{equation}
  \includegraphics[scale=0.7]{problem_13}
\end{document}